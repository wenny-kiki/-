\documentclass{article}
\usepackage[UTF8]{ctex}
\usepackage{geometry}
\usepackage{multirow}
\usepackage{natbib}
\geometry{left=3.18cm,right=3.18cm,top=2.54cm,bottom=2.54cm}
\usepackage{graphicx}
\pagestyle{plain}	
\usepackage{setspace}
\usepackage{enumerate}
\usepackage{caption2}
\usepackage{datetime} %日期
\renewcommand{\today}{\number\year 年 \number\month 月 \number\day 日}
\renewcommand{\captionlabelfont}{\small}
\renewcommand{\captionfont}{\small}
\begin{document}

\begin{figure}
    \centering
    \includegraphics[width=8cm]{upc.png}

    \label{figupc}
\end{figure}

	\begin{center}
		\quad \\
		\quad \\
		\heiti \fontsize{45}{17} \quad \quad \quad 
		\vskip 1.5cm
		\heiti \zihao{2} 《计算科学导论》个人职业规划
	\end{center}
	\vskip 2.0cm
		
	\begin{quotation}
% 	\begin{center}
		\doublespacing
		
        \zihao{4}\par\setlength\parindent{7em}
		\quad 

		学生姓名:\underline{\qquad  李坤桐 \qquad \qquad}

		学\hspace{0.61cm} 号:\underline{\qquad 1801050125\qquad}
		
		专业班级:\underline{\qquad 计科1801 \qquad  }
		
        学\hspace{0.61cm} 院:\underline{计算机科学与技术学院}
% 	\end{center}
		\vskip 1.5cm
		\centering
		\begin{table}[h]
            \centering 
            \zihao{4}
            \begin{tabular}{|c|c|c|c|c|c|c|c|c|}
            % 这里的rl 与表格对应可以看到,姓名是r,右对齐的;学号是l,左对齐的;若想居中,使用c关键字。
                \hline
                \multicolumn{5}{|c|}{分项评价} &\multicolumn{2}{c|}{整体评价}  & 总    分 & 评 阅 教 师\\
                \hline
                自我 & 环境 & 职业 & 实施 & 评估与 & 完整性 & 可行性 &\multirow{2}*{} &\multirow{2}*{}\\
                分析& 分析& 定位 & 方案 & 调整 & 20\% & 20\% & ~&~ \\\            
                10\% & 10\% & 15\% & 15\% & 10\% & &  &~ &~\\
                \cline{1-7} 
                & & & & & & & ~&~ \\
                & & & & & & & ~&~ \\
                \hline      
            \end{tabular}
        \end{table}
		\vskip 2cm
		\today
	\end{quotation}

\thispagestyle{empty}
\newpage
\setcounter{page}{1}
% 在这之前是封面,在这之后是正文
\section{自我分析}
	\par
	\par
\subsection{自然条件}
性别女,年龄21岁,家住山东省潍坊市。身体硬朗,体脂和BMI长年都在正常范围内,20年8月刚做过一次全面体检,非常健康。经常锻炼,注重养生,发际线目前还不是很危险,视力略有下降,但在正常范围内,肩颈和腰背变差了非常多,未来多多注意,为打持久战建立良好基础。\par
\subsection{性格分析}
自我感觉:善于独立思考,注重团体力量,善解人意,做事全力以赴,有耐性,刻苦,实际而热情,意志坚定;但沟通能力不足,不善表达。\par
性格测试:对别人的情绪敏感,能理解、体会别人的心情,善于安慰、鼓励别人;对文字、语言敏感;善于分析、总结;善于从整体上把握事物;能理解复杂的理论概念,善于将事情概念化,善于从中推断出原则;擅长策略性思维。但是,有仅仅凭个人的好恶或价值观来决定事情,并希望别人也以同样的角度或标准来处理问题的倾向;有时他们心里老想着别人的问题,可能会过于陷于其中,以至于被其困扰;有时容易将别人或事情理想化,不够实际;不是特别善于管束和批评他人,尽管常常自我批评;有时会为了和睦而牺牲自己的意见或利益;比较容易动感情,情绪波动较大。
\par
\subsection{教育与学习经历}
高中就读于潍坊第一中学,学风开放,容·雅文化影响终生,在教育压力很大的山东省,我在学校感受到了浓浓的人文情怀,固定的阅读课、兴趣课、体育课,直到高三也只是削减时间而没有取消,身边的人都各有特色、背景,全面发展又特长明显,我也早早的明白,人是多元的,人与人是不一样的,但我依然喜欢,在这样优秀的环境里生活,可以潜移默化的影响我很多。\par
在中国石油大学华东已经学习了完整的两年,大一我的专业是地理信息科学,大二我的专业是计算机科学与技术,其实都是计算机相关的专业,自己也是在大学中才第一次接触,并且在大一学习过程中发现喜欢编程,慎重考虑后转到了现在的专业。大一的学习经历是轻松自然的,尽管一些基础课程比较难,但好在努力就会有收获,成绩最后排名专业第4,参加了很多社团活动,综合评价排名专业第2。大二的学习经历就相对困难了很多,因为性格原因,很难和新同学熟识,也不住在一起,自然交流的机会很少,对各方面都产生了一定影响,经常会觉得信息不对等,很吃力,只能闷头做事;最后大二的成绩专业排名第10,我是有点失望和迷茫的,陷入了学习不怎么样,专业能力又不强的困境。
\par
\subsection{工作与社会阅历}
在高三毕业暑假有过20天的便利店收银工作经历,我觉得其实是有趣的,没有想象中做几天就无聊了的情况,每天可以观察到很多人。在大学,断断续续的做家教兼职,有通过中介认识的,有学长朋友介绍的,发现和家长沟通比教学生难对付多了。大一暑假参加了青岛国际电影节的志愿者活动,最初是为了见明星,但在过程中收获了更多,或者说见识到了很多,自己被分到了新媒体部门,遇到的更多是记者,通过一个新的领域看大型演出的整个历程,非常有趣。遗憾的是直到现在,自己也没有过专业相关的实习经历,希望能尽快填补遗憾。\par
\subsection{知识、技能与经验}
参与了大创项目,在第一次阶段测评后被立项为国家级大创项目,参加计算机设计大赛获得国赛三等奖省赛一等奖,但遗憾的是基于项目没有产出论文和软件专利。\par
\subsection{兴趣爱好与特长}
爱好阅读,早些年喜欢泡在书店、图书馆看各类小说、漫画,近几年更加注重看有所求,会读一些功利性较强的书和补一些经典书籍;体育方面兴趣广泛,擅长网球、乒乓球、羽毛球等;系统学习过吉他、素描,尽管并不擅长和喜欢,但可以说明自己小时候也是有过兴趣培养经历的。\par
\section{环境分析}
\par
\par
\subsection{社会环境分析}
我们现在面临一个非常好的宏观环境,中国社会安定,政治稳定,法制建设不断完善,文化繁荣自由,尖端技术、高新技术突飞猛进。经济发展迅速,与全球一体化接轨,在全球经济一体化环境中扮演重要角色,有大批的国外企业进入中国市场,中国的企业也走出了国门。就业形势虽然严峻,但也充满了机遇与挑战。\par
\subsection{家庭环境分析}
家庭经济状况一般,还算良好;家庭成员文化程度都不是很高,都是勤劳的打工人,家庭和睦,身体也都比较健康;父母期望不大,所以从小给我的压力不大,无论自己在学习方面失利或突然很好,父母都不会有特别大的波动,给了我很大的空间和强有力的支持。\par
\subsection{职业环境分析}
全球现处于第三次工业革命,IT行业在其中举足轻重;且现在也是“三步走”战略和“新三步走”战略的关键时期。这一时期一定会有很大的机遇和契机,我们正处于这一伟大的时机。从市场总体发展情况来看,伴随着中国逐渐成为全球IT企业关注的重心,中国IT业市场的竞争日趋激烈。中国不但已经成为全球重要的IT制造中心,同时也逐渐步入全球IT研发中心的角色。良好的国内经济环境使中国IT行业市场继续保持平稳的增长,蓬勃发展的中国软件与IT服务市场吸引了众多新的进入者。现在我国的IT行业人才不足且有严重的结构性的失衡,但其中也更重要的是高技术人员的需求。社会需要的更多的是高技术性的IT人才,用人单位更是提高这方面的门槛。现在计算机已经得到了极大的普及,各高校都很重视这方面的培养,企业也重视培养。在大学生就业形势危急的情形下,IT行业也是日趋激烈。展望未来,中国国民经济继续保持稳步地增长,中国行业和企业信息化建设进一步深化,消费者消费水平逐步提高,中国IT行业市场具有良好的发展前景。机遇与挑战并存,大浪淘沙,唯有具备实力的人才能脱颖而出。\par


\subsection{地域与人际环境分析}
关于未来工作的城市,我的第一选择是杭州。杭州气候很好,温暖湿润,四季分明,光照充足,雨量丰沛,感觉适宜人居住;文化比较开放,是旅游名城,有烟雨缥缈的湖光山色,城市比较富裕,周边城市发展也不错。并且杭州有许多互联网公司,不仅有互联网巨头,也有大量的中小型互联网公司在梦想小镇、云栖小镇等,有比较大的选择空间;杭州落户较易,房价远低于北上广深,整体氛围不那么浮躁,政府也非常重视互联网并大力投入;杭州高校实力远不及北京,人才缺口比较大,所以需要从外地引进人才,有数据显示,杭州的人均面邀、入职率以及薪资相比北上广深都是有一定优势的。长三角是中国经济的主力,所以我很看好杭州,希望未来我也能被看好。自己在杭州并没有什么人脉,其实在杭州上学的同学也不是很多,但我觉得未来会有不少朋友会去杭州工作发展。\par




\section{职业定位}
\par
\par

\subsection{行业领域定位与理由}
意向工作地点是软件外企的中国分公司或互联网民企,希望自己能趁着年轻在行业市场的前端试探和活跃,并且如果能入职,这类公司一般起薪较高,职业培养良好,上升快;当然我也能接受加班多,不稳定的危险因素。
领域主要意向是技术岗,也是互联网比较热门的职业,职位需求比较多,自己也比较感兴趣,能够坚持做下去。
\par
\subsection{职业岗位起点定位与理由}
起点:开发工程师,可以真正实现自己所学,在真实项目中验证自己的能力,提升综合素质和积累经验;而且自己在学生时期项目经验较少,能力也不算出众,所以选择从基础码农做起是比较好的选择,后续积极表现,争取提升。\par
\subsection{职业目标与可行性分析}
\par
\par 
\begin{enumerate}[(1)]
	\item 短期目标(大学4年)
	\par大学只剩一年半的时间,大三下学期,通过英语六级,扎实基础,了解就业形势,做好就业准备,争取获得暑期实习的机会。大四上学期,争取获得保研机会,若没有获得就认真准备考研。大四下学期,认真完成毕业设计,争取就业机会,如果考研失败可以考虑二战。
	\item 中长期目标(5-10年)
	\par一定要获得研究生学位。在研究生阶段努力加强综合能力,毕业后,找到一份不错的工作,积极表现,争取提升的机会,取得一定的工作成绩。实现经济独立,并且有积累存款的能力。
	\item 可行性分析
	\par短期目标可行性还是比较高的,中长期目标还需要后期再视情况做仔细规划。
\end{enumerate}



\section{实施方案}
面对现实,我必须首先要好好学习,扎实专业课程,要有过硬的专业知识。在大三下学期一定要认真学英语,通过英语六级,掌握一定的计算机英语,并且满足保研要求。对我来说,对研究生和就业都是特别美好的,但我一定要争取一下读研(无论是考研还是幸运的保研),完成接触更高段位学习内容的梦想。同时对标职业定位不断提高自己的综合素质,逐步了解社会、企业对当代大学生的要求,不至于到毕业时再一脸茫然。学习工作之余要坚持兴趣爱好,与人交流,保证身心健康。\par
\section{评估与调整}
这次职业规划设计,较详细、合理的规划了自己的职业。分时间、任务,分步骤地完成对目标的追求。倘若有哪个时间段没有完成目标,一定要抓紧在接下去的时间段补上。\par 
\subsection{评估时间}
每学期一次\par
\subsection{评估内容}
看是否收获了可以看到的成果,如果有则证明目标基本完成。同时考虑,自己的能力是否得到提升,这是实质的目标。综合以上两点,成果和能力,对已实现的总结经验,对未完成的分析原因并做出调整。\par
\subsection{调整原则}
既然已经在前期做好了规划,最应该全力以赴将之付诸行动;当然也需要随时了解情况,当问题出现时,作出适当的选择,尽量只做小的调整,大的目标要坚定。\par
假设未来自己真的由于种种艰难不能完成理想,就回老家从事教育行业,已经通过了教师资格证的两门笔试,计划在毕业前将剩下的一门笔试和一门面试通过,单看性格和专业基础知识,教师还是比较适合自己的。
\par




\end{document}
